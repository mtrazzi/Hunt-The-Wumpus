\documentclass[a4paper, 10pt]{article}

\usepackage[french]{babel}
\usepackage[utf8]{inputenc}
\usepackage[T1]{fontenc}
\usepackage[top=3cm, bottom=3cm, left=3cm, right=3cm]{geometry}
\usepackage{lmodern, amsmath, amssymb, mathrsfs, graphicx, listings, tabularx, color, pgfplots, pgfplotstable, booktabs, titling, authblk, parskip, pgfplots, csquotes}
\usepackage{hyperref}
\usepackage[backend=biber, url=false, isbn=false, sorting=none, natbib]{biblatex}
\usepackage[labelfont=sc]{caption}
\usepackage[bottom]{footmisc}
\setlength\parindent{1cm}
\MakeOuterQuote{"}
\hypersetup{colorlinks, linkcolor=black, urlcolor=black}
\pgfplotsset{compat=1.15}
\addbibresource{biblio.bib}
\bibliography{biblio}
\newcommand{\HRule}{\rule{\linewidth}{0.5mm}}
\newcolumntype{Y}{>{\centering\arraybackslash}X}
\newcommand{\Var}{\mathrm{Var}}
\DeclareMathOperator*{\argmax}{arg\,max}

%%% MICHAEL ADDED

\usepackage{titlesec}
\usepackage{enumitem}


\setcounter{secnumdepth}{4}

\titleformat{\paragraph}
{\normalfont\normalsize\bfseries}{\theparagraph}{1em}{}
\titlespacing*{\paragraph}
{0pt}{3.25ex plus 1ex minus .2ex}{1.5ex plus .2ex}

%%%

\begin{document}

\begin{titlepage}
\begin{center}
~\\[1cm]
\Large Faculté des Sciences et Ingénierie\\Sorbonne Université\\[3.5cm]
\HRule 
\\[0.4cm]{\huge \bfseries Projet - Wumpus Multi-agent\\[0.4cm]}
\HRule \\[1cm] 
\Large \textsc{Michaël Trazzi, Michaël Aidan} \\[0.1cm]
\normalsize Sous la supervision d'\textsc{Aurélie Beynier, Nicolas Maudet} et \textsc{Cedric Herpson}\\[2cm]
\Large Master 1 Informatique, spécialité ANDROIDE\\Année $2017-2018$, Semestre 2 \\[4cm]
\includegraphics[scale=0.3]{logo.png}
\end{center}
\end{titlepage}

\tableofcontents

\newpage
\section{Introduction}

Pour ce projet nous avons été amenés a implémenter un Système Multi-Agent pour résoudre un problème en apparence simple : la collecte de trésors sur un graphe. Cependant, l'existence d'une perception limitée des agents, et la présence d'un terrifiant "Wumpus" sur la carte a rendu difficile l'implémentation d'une solution satisfaisante.

\subsection{Présentation du probleme}

Il y a 4 types d'agents: les Explorateurs, pouvant uniquement se déplacer, les Collecteurs, capables de ramasser les trésors de leur type jusqu'a remplir leur sac, les agents Silo, pouvant accumuler sans limite les trésors ramassés par les agents Collecteur, et l'agent "Wumpus", se mouvant aléatoirement sur la carte en déplaçant les trésors.

Les seuls agents dont nous pouvons modifier le comportement sont les agents Explorateurs, Collecteurs et Silo. Ces agents vivent dans un environnement \verb|JADE|, environnement de développement pour des systèmes multi-agents en \verb|Java|. Ils exécutent des comportements (ou \textit{behavior}) séquentiellement, dans un ordre pré-déterminé, a chaque fois qu'un processus les réveille.

En Intelligence Artificielle, un agent est n'importe quel objet percevant son environnement par des \textit{senseurs}, et agissant sur le monde a travers des \textit{effecteurs}.  Le Wumpus n'ayant pas de comportement complexe (il ne prend pas de décision mais se contente de faire des mouvements aléatoires), nous désignerons par \textit{agent} les trois autres agents (Explorateurs, Collecteurs et Silo). Ces trois agents (qu'on a été amené a implémenter) disposaient de senseurs et effecteurs relativement précaires.

\begin{itemize}
    \item \textbf{Senseurs}
        \begin{itemize}
            \item \underline{Observation} Un agent peut exécuter la méthode \verb|observe|, qui retourne une liste d'attributs des noeuds adjacents au noeud ou se trouve l'agent. L'agent sait donc quels noeuds sont adjacents, et si ces noeuds contiennent des trésors. Ils n'a pas \textit{a priori} connaissance de la présence d'un autre agent sur un de ces noeuds.
            \item \underline{Réception} de message.
        \end{itemize}
        
    
    \item \textbf{Effecteurs}
            \begin{itemize}
            \item \underline{Mouvement}
            \item \underline{Collecte de trésor}
            \item \underline{Transmission de trésor}
            \item \underline{Emission de message}
        \end{itemize}
\end{itemize}

\subsection{Problématiques}

Ou on explique les problématiques résultantes du probleme presenté ci-dessus.

\subsection{Ressources}

Notre implémentation est disponible sur \verb|github| a l'adresse suivante : \url{https://github.com/mtrazzi/Hunt-The-Wumpus}. De plus, nous avons rédigé un article, publié en ligne, expliquant en détail les difficultées liées aux Systèmes Multi-agents, et en particulier pourquoi implémenter des protocoles Multi-Agents pouvait s'avérer bien plus complexe que simplement faire de l'apprentissage machine supervisé. Cet article de vulgarisation est hebergé sur \verb|Medium| : \url{https://hackernoon.com/why-coding-multi-agent-systems-is-hard-2064e93e29bb}.

\section{Analyse des algorithmes}

Introduction générale des différents algorithmes.

\subsection{Exploration}

\subsubsection{Principe}

\subsubsection{Avantages/inconvénients}

\paragraph{Forces}

\paragraph{Limites}

\subsubsection{Complexité}

\paragraph{Critere d'arret}

\paragraph{Temps}

\paragraph{Mémoire}

\paragraph{Communication}

\paragraph{Optimalité}

\subsection{Communication}

\subsubsection{Principe}

\subsubsection{Avantages/inconvénients}

\paragraph{Forces}

\paragraph{Limites}

\subsubsection{Complexité}

\paragraph{Critere d'arret}

\paragraph{Temps}

\paragraph{Mémoire}

\paragraph{Communication}

\paragraph{Optimalité}

\subsection{Interblocages}

\subsubsection{Principe}

\subsubsection{Avantages/inconvénients}

\paragraph{Forces}

\paragraph{Limites}

\subsubsection{Complexité}

\paragraph{Critere d'arret}

\paragraph{Temps}

\paragraph{Mémoire}

\paragraph{Communication}

\paragraph{Optimalité}

\subsection{Ramassage des Trésors}

\subsubsection{Principe}

\subsubsection{Avantages/inconvénients}

\paragraph{Forces}

\paragraph{Limites}

\subsubsection{Complexité}

\paragraph{Critere d'arret}

\paragraph{Temps}

\paragraph{Mémoire}

\paragraph{Communication}

\paragraph{Optimalité}

\subsection{Coordination}

\subsubsection{Principe}

\subsubsection{Avantages/inconvénients}

\paragraph{Forces}

\paragraph{Limites}

\subsubsection{Complexité}

\paragraph{Critere d'arret}

\paragraph{Temps}

\paragraph{Mémoire}

\paragraph{Communication}

\paragraph{Optimalité}

\subsection{Migration}

\subsubsection{Principe}

\subsubsection{Avantages/inconvénients}

\paragraph{Forces}

\paragraph{Limites}

\subsubsection{Complexité}

\paragraph{Critere d'arret}

\paragraph{Temps}

\paragraph{Mémoire}

\paragraph{Communication}

\paragraph{Optimalité}

\newpage
\section{Conclusion}

\subsection{Synthese}

\subsection{Regard critique sur notre travail}

\subsection{Extensions et améliorations possibles}

\end{document}